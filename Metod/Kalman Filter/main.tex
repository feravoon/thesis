\documentclass{article}
\usepackage[utf8]{inputenc}
\usepackage[%
    left=1in,%
    right=1in,%
    top=1.0in,%
    bottom=1.0in,%
    paperheight=297mm,%
    paperwidth=210mm%
]{geometry}%
\setlength{\parskip}{\baselineskip}%
\setlength{\parindent}{0pt}%

\title{Kalman Filter}
\author{Ahmet Çakıroğlu}
\date{}

\usepackage{natbib}
\usepackage{graphicx}

\begin{document}

\maketitle


%\begin{equation}
%\hat{X}_{k} = K_{k}.Z_{k}+(1-K_{k}).\hat{X}_{k-1}
%\end{equation}
Kalman filter is an estimator which estimates the state of a linear system. It uses measurements that are generated as a linear function of the system state but corrupted by additive Gaussian noise. Kalman filter employs two procedures called time update(prediction) and measurement update(correction). Let $x_k$ be the state of the system at time step $k$, then state transition at each time step can be written as a linear combination of the previous state, a control signal and noise:
\begin{equation}
x_k = Ax_{k-1}+Bu_k+w_{k}
\end{equation}
where $A$ is a matrix called state transition matrix, $u_k$ is a control signal assuming the system is controlled by another system, $B$ is a matrix representing the control model and $w_k$ is the process noise. Measurements from the system can be modelled as a linear transform of the state with additive Gaussian noise:
\begin{equation}
z_k = Hx_k+v_k
\end{equation}
where $z_k$ is the measurement, $H$ is the measurement model and $v_k$ is the measurement noise. $A$, $B$ and $H$ matrices of the system are constant and must be known or correctly modelled for Kalman filter to work correctly.
\subsection*{Time Update (Prediction)}
Kalman filter makes a prediction about the state of the system in next iteration. The prediction is simply made by using the state transition model:
\begin{equation}
\hat{x}_{k}^{-} = A\hat{x}_{k-1}+Bu_{k}
\end{equation}
where $\hat{x}_{k}^{-}$ is the state prediction or the state prior. Since it is an estimator, Kalman filter has a state estimation covariance matrix denoted as $P_k$. Prediction step also makes a prediction about what estimation covariance will be at next iteration, in other words it projects ahead the error covariance for one step:
\begin{equation}
P_{k}^{-} = AP_{k-1}A^{T}+Q
\end{equation}
where $Q$ is the covariance matrix of the process noise $w_k$.

\subsection*{Measurement Update (Correction)}
In correction equations, new measurement about the system is used to update previously predicted state estimate. Firstly, Kalman gain, denoted as $K_k$ is calculated. Kalman gain is a weighting factor that determines how much the new measurement affects the new estimate of the state. Kalman gain is calculated by using the state estimation covariance matrix $P_k$ and measurement covariance matrix $R$:
\begin{equation}
K_{k} = P^{-}_{k}H^{T}(HP^{-}_{k}H^{T}+R)^{-1}
\end{equation}
Note that $R$ is a constant user defined value, so one must correctly define how noisy the measurements would be. State prediction is updated with a value called the measurement prediction error or the residual.
\begin{equation}
\hat{x}_{k} = \hat{x}_{k}^{-} + K_{k}(z_{k}-H\hat{x}_{k}^{-})
\end{equation}
$H\hat{x}_{k}^{-}$ is what would the measurement be if the state prediction is correct. Therefore $(z_{k}-H\hat{x}_{k}^{-})$ is the difference between the measurement $z_k$ and predicted measurement $H\hat{x}_{k}^{-}$, so it is the measurement residual. State estimate is corrected by the measurement residual weighted by Kalman gain. Correction step also corrects the state estimation covariance prediction $P_k^{-}$ using calculated Kalman gain:
\begin{equation}
P_{k} = (I-K_{k}H)P^{-}_{k}
\end{equation}

Prediction and correction procedures run at every iteration respectively. Outputs of every iteration are used for inputs to next iteration.
\bibliographystyle{unsrt}
\bibliography{references}
\end{document}