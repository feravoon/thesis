\documentclass{article}
\usepackage[utf8]{inputenc}

\usepackage{amsmath}
\usepackage{amsthm}
\usepackage{amssymb}

\usepackage[%
    left=1in,%
    right=1in,%
    top=1.0in,%
    bottom=1.0in,%
    paperheight=297mm,%
    paperwidth=210mm%
]{geometry}%
\setlength{\parskip}{\baselineskip}%
\setlength{\parindent}{0pt}%

\newcommand{\btheta}{\boldsymbol{\theta}}


\title{Joint Probabilistic Data Association Filter}
\author{Ahmet Çakıroğlu}
\date{}

\usepackage{natbib}
\usepackage{graphicx}

\begin{document}

\maketitle

JPDAF defines a joint association event with following:
\begin{equation}
\btheta=\bigcap_{j=1}^{m_k}{\theta{jt_j}}
\end{equation}
where
\begin{gather}\nonumber
{\theta{jt}}\triangleq \text{\{measurement $j$ originated from target $t$\},}\\
j=1,...,m_k;~~~~~t=0,1,...,T
\end{gather}

$T$ is the number of targets being tracked and $m_k$ is the number of measurements in time step $k$ (latest time step). Instead of considering all joint association events, JPDAF uses a validation matrix to eliminate association events with negligible probability in order to reduce computational complexity. If a measurement $j$ is outside of the validation gate of target $t$, association probability of the event $\theta_{jt}$ is considered negligible. JPDAF defines a validation matrix as follows:
\begin{gather}
\boldsymbol{\Omega}\triangleq[\omega_{jt}],~~~~~~j=1,...,m_k;~~~~~t=0,1,...,T
\end{gather}
where $\omega_{jt}$ is a binary value that indicates if measurement $j$ is in the validation gate of target $t$.

A joint association event is called \textit{feasable} if a measurement can only be originated from one source, i.e.,
\begin{gather}
\sum_{t=0}^{T}\hat{\omega}_{jt}(\btheta)=1,~~~~~~j=1,...,m_k
\end{gather}
and no more than one measurement can originate from a target, i.e.,
\begin{gather}
\delta_t(\btheta)\triangleq\sum_{j=1}^{m_k}\hat{\omega}_{jt}(\btheta)\leq1,~~~~~~t=1,...,T
\end{gather}
$\delta_t(\btheta)$ is called \textit{target detection indicator}. It indicates that target $t$ is associated with a measurement in the joint event $\btheta$. For the convenience of the main equation of JPDAF, a binary \textit{measurement association indicator} $\tau_j(\btheta)$ is also defined. It indicates that measurement $j$ is associated with a target.
\begin{gather}
\tau_j(\btheta)\triangleq\sum_{t=1}^{T}\hat{\omega}_{jt}(\btheta),~~~~~~j=1,...,m_k
\end{gather}

\begin{gather}
P\{\btheta(k)|Z^k\}=\frac{1}{c} \dfrac{\phi!}{V^\phi} \prod_{j=1}^{m_k}\Big[N_{t_j}[\bold{z}_j(k)]\Big]^{\tau_j} \prod_{t=1}^T (P^t_D)^{\delta_t}(1-P^t_D)^{1-\delta_t}
\end{gather}
\bibliographystyle{unsrt}
\bibliography{references}
\end{document}