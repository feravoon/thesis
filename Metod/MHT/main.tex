\documentclass{article}
\usepackage[utf8]{inputenc}

\usepackage{amsmath}
\usepackage{amsthm}
\usepackage{amssymb}

\usepackage[%
    left=1in,%
    right=1in,%
    top=1.0in,%
    bottom=1.0in,%
    paperheight=297mm,%
    paperwidth=210mm%
]{geometry}%
\setlength{\parskip}{\baselineskip}%
\setlength{\parindent}{0pt}%

\newcommand{\btheta}{\boldsymbol{\theta}}


\title{Multiple Hypothesis Filter}
\author{Ahmet Çakıroğlu}
\date{}

\usepackage{natbib}
\usepackage{graphicx}

\begin{document}

\maketitle

Multiple hypothesis filter is an algorithm to resolve measurement-to-track association problem in multi-target tracking systems. Unlike joint probabilistic data association filter (JPDAF) which only considers latest set of measurements and already established tracks; multiple hypothesis filter generates and holds hypotheses about all possible association decisions from the beginning, including initiation of a new track. In each iteration, association decision is chosen as the hypothesis with the highest probability while other hypotheses are retained and improved. This prevents an incorrect association to cause a complete mix up of tracks.

Let $\boldsymbol{\Omega}^k$ be the set of all association hypotheses up to time step $k$. $\boldsymbol{\Omega}^k$ consists of joint cumulative events. A joint cumulative event $\Theta^{k,l}$ at time step $k$ can be defines as follows
\begin{gather}
\Theta^{k,l} = \{ \Theta^{k-1,s}, \btheta(k) \}
\end{gather}
where $\btheta(k)$ is a joint event in time k which consists of $\tau$ number of measurements originated from an existing track, $v$ number of measurements originated from new targets and $\phi$ number of measurements that are false alarms. Following indicators are defined for latest set of measurements $\boldsymbol{z}_i(k)$, $i=1...m_k$
\begin{align}
\tau_i = \tau_i[\btheta(k)] \triangleq & \begin{cases}
    1, & \text{if $\boldsymbol{z}_i(k)$ originated from an existing track}\\
    0, & \text{otherwise}\\
  \end{cases}\\ \notag\\
v_i = v_i[\btheta(k)] \triangleq & \begin{cases}
    1, & \text{if $\boldsymbol{z}_i(k)$ originated from a new target}\\
    0, & \text{otherwise}\\
  \end{cases}\\ \notag\\
  \delta_t = \delta_t[\btheta(k)] \triangleq & \begin{cases}
    1, & \text{if track $t$ is detected at time step $k$}\\
    0, & \text{otherwise}\\
  \end{cases}
\end{align}
Joint cumulative events in $\boldsymbol{\Omega}^{k-1}$ is augmented with feasible joint association events in time srep $k$ to create $\boldsymbol{\Omega}^k$. Conditional probability of each cumulative event is calculated as follows
\begin{gather}
P\{\Theta^{k,l}|Z^k\} = \frac{1}{c} \frac{\phi!v!}{m_k V^{\phi v}} \mu_F(\phi) \mu_N(v) \prod_{i=1}^{m_k}\Big[N_{t_i}[\bold{z}_i(k)]\Big]^{\tau_i} \prod_{t=1}^T (P^t_D)^{\delta_t}(1-P^t_D)^{1-\delta_t} P\{\Theta^{k-1,s}|Z^{k-1}\}
\end{gather}
where $Z^k$ is the set of all measurements up to time step $k$, $\bold{z}_i(k)$ is an individual measurement, $c$ is a normalization constant, $V$ is the surveillance region volume, $N_{t_j}$ is the multivariate normal distribution function generated with the covariance matrix produced by the state estimator of target $t_i$. $P^t_D$ is the probability of detection of target $t$. $\mu_F(\phi)$ and  $\mu_N(v)$ are the probability mass functions of number of false alarms and number of new targets. After calculating the probabilities of all joint cumulative events in $\boldsymbol{\Omega}^k$, MHF chooses joint cumulative event with the highest probability as the association decision.

Number of hypotheses grow exponentially as new measurements arrive. MHT performs hypothesis reduction techniques to keep the number of hypotheses at a reasonable level. Hypotheses with negligible probabilities are eliminated and hypotheses which have similar associations are merged.+-
\bibliographystyle{unsrt}
\bibliography{references}
\end{document}